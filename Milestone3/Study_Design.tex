%"Study_Design.tex", by Sean Soderman
%Outlines all the sorts of things we wish to consider for our user study

\documentclass[11pt]{extarticle}
\usepackage{times}

\begin{document}
\title{Study Design}
\author{Sean Soderman\\ 
        \and
        Dylan Soderman
}
\date{\today}
\maketitle

\section*{The Why} %NOTE: Flip this with the Concerns
The purpose for this usability study is to find out how comfortable people are with navigating the website. In 
particular, we are evaluting how quickly our users can accomplish browsing-based tasks, as well as 
\subsection*{Effects}
We hope to fortify our website to better accomodate the needs of our users. In particular, we hope to minimize the
difficulty of using the site's features, such as using the search function or uploading images. The thresholds
we are defining that will let us know when things must be improved/changes are if people take longer than the
allotted time (5 seconds) to complete a task. For blind users, this threshold will be increased by a factor of
three. This may prompt us to improve the site's code using ARIA to make each element's
purpose completely obvious to the user, regardless of sightedness.
\subsection*{Flexibility \textit{(or how much can be changed)}}
The less central elements of the website, such as the look of the search results can be modified to be more colorful
and readable. The user's profile page can also accomodate pictures for each review that user has written, so that
there is no ambiguity as to which game they wrote that review for. The search engine could be improved to allow
searching by any paramater (number of reviews, by date published, or alphabetically). Also, the review page can
include the title and or image of the game the user is reviewing so that if they take a break in the middle of
reviewing it, they will not forget what game they were reviewing. The icons on the user's profile page can
be exchanged for more intuitive ones using more traditional symbols. The about me can also be modified to
retain its text until the user explicitly deletes it.
\section*{The What: Concerns}
We believe that our search engine may prove insufficient for savvier users' needs. This is since the engine
only matches its input to the name of a game. It does not search for review titles, users, or comments made by
a commenter. We are also slightly worried that our users may not understand the meaning of icons on their profile
page, and that they may find the way that the text within their about me being deleted each time they edit
it annoying. There is also the issue of there being no indication of what game the user is reviewing on the
review page. Finally, we don't feel confident about the usability of our game lists, for they are ordered
by release date rather than alphabetically.
\subsection*{Goals}
   \begin{itemize}
   \item The user should be able to search for a game in less that five seconds.
   \item The user should be able to find a game within a list of games for a certain console within 20 seconds.
   \item The user will be able to edit their about me within 15 seconds.
   \item The user will notify the test-giver how easy it was to modify their profile.
   \end{itemize}
\subsection*{Tasks}
   \begin{itemize}
   \item Look for the game Little Big Planet 2.
   \item Register an account and set up your profile.
   \item Look for a  game with more than 1 review.
   \item Write about a game you recognize.
   \end{itemize}
\subsection*{Scenarios}
   \subsubsection*{Create a profile}
      You have just heard about the new website for all your game review needs, DjangoFett, and wanted
      to start up a profile on the website. It would be most satisfying to have a fully customized profile.
      You would like to be one of the first people to write a review on this
      website. You have in mind a recent game you recognize.
   \subsubsection*{Finding games}
      Upon finishing your review, you wanted to check if the game Little Big Planet 2 had many reviews, since it is
      rather popular. You are now wondering if there is a game that has at least more than 1 review
      on the website. You let one of the reviewers know whether you agree or disagree with an up/down vote
      and comment.
\subsection*{Measurements}
   \begin{itemize}
   \item The amount of time a user takes to do a task will be recorded. 
   \item Blind user's amount of mistakes will be tallied.
   \item Whether a user used the search function or one of the game lists.
   \item A 1-5 point value denoting the comfort level of the user as they wrote their review.
   \end{itemize}
\section*{How: Test Methods}
   Since this website could be considered to be in an infant state of development, a diagnostic test
   enacted upon the above concerns would be most appropriate.
\end{document}
